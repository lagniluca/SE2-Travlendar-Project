\documentclass[a4paper]{book}

\usepackage[T1]{fontenc}
\usepackage[utf8]{inputenc}
\usepackage[english]{babel}
\usepackage{blindtext}
\usepackage{amsmath}

\begin{document}

\date{2017/10/09}
\author{Bolshakova Liubov\\ Campagnoli Chiara\\ Lagni Luca}
\title{RASD}
\frontmatter                            % only in book class (roman page #s)
\maketitle                              % Print title page.
\tableofcontents                        % Print table of contents
\mainmatter   

\part{Introduction}

\chapter{Pourpouses}
In this chapter we will deline all the goals that we have extract from our text and the actors that take part
in the system and the environment of our application.

\section{Goals}

\begin{itemize}
\item (G01): The software should allow the user to register
\item (G02): Every user can decide what path to do, inside a city or a region.
\item (G03): Every user can decide what mean of transports would take, according to the ones provided by the software.
\item (G04): Every user can decide the range of time to reserve for breaks. 
\item (G05): The system must comunicate to the user that a path (selected by this one) is not reachable or it's out of time.
\item (G06): The system must provide all the possible path that can be taken by a user, according to his/her needed.
\item (G07): The system must provide the most optimized and suitable solution , according to the user constraint, as first suggested path.
\item (G08): The system must provide informations about problems/strikes for all the means of transports included in the software.
\item (G09): The system must warning the user of weather problems concerning some means of transport.
\item (G10): The system must provide a way to permit to a single user to buy a ticket for public transports.
\item (G11): The system must provide the nearest location of a bike provided by a bike sharing service provider included in the software.
\item (G12): The system must provide the nearest location of a car provided by a car shering service provider included on the software.
\item (G13): The system must avoid overlaps in user's scheduled travels.
\item (G14): The system must allow the user to create different types of meeting (at least, business meetings).
\item (G15): The system must warning the user about upcoming meetings.
\item (G16): The system must allow the user to delete a path.
\item (G17): The system must provide an alternative path in case of problems along the path selected.
\item (G18): The system must interface for wearable devices, providing maps or , at least , messagges.
\item (G19): The software should show to each user, if possible, the the combination of means of transport that minimize the carbon footprint, according to the path selected and the required time.

\end{itemize}

\section{Actors}
\begin{itemize}

\item (A01): User - A generic actor that uses the application
\item (A02): Visitor - A user that it's not registred yet
\item (A03): Client - A Registred user
\item (A04): Techical Support - Actor that aid the user in case of techical problems with the app.

\end{itemize}

We have done a distinction between Client and Visitor because we think that , one day, \\
it will be usefull , for our application, to evolved and a special treatment for making the \\
user fidelity stronger will be desiderable or required.\\

Anyway, at this stage , there are no difference in features between a visitor and a client\\
so, if not necessessary, we will refer to them as users.

\section{Agents}

This application purpouse is to aim clients for short travels, so we limited our means of transport in:

\begin{itemize}

\item (M01): Feet 
\item (M02): Personal bike 
\item (M03): Personal car
\item (M04): Other autonomous personal means of transport
\item (M05): Other non autonomous personal means of transport
\item (M06): Bike provided by a bike sharing provider, if available
\item (M07): Car provided by a car sharing provider, if available
\item (M08): Other non autonomoys means of transport???
\item (M09): Other autonomous means of transport
\item (M10): Trains
\item (M11): Trams
\item (M12): Bus 
\item (M13): Taxi
\item (M14): Boat (for cities like Venice).

\end{itemize}

We can exclude planes, because of the nature of the travels considered.\\

Other autonomous personal means of transport can be everything that has an engine like motorcycles , quad, segway ...\\

Other non autonomous personal means of transport can be everything, used for move from one place to another, that hasn't an engine like rollerblade, skateboards ...\\

Other non autonomous means of transport can be tandems ... \\

Other autonomous means of transport can be rented vehicles like motorcycles , cars not provided by a carsharing but also hichkicking and so on ... \\

For all the "other [...] means of transport" we don't provide a specialized definition, we only focus of maximum speed, euro class and possible special access/limitations (like for veichles disposed for disable people), that we assume mandatory (the client must provided this kind of informations if he intended to use this kind of veichles), if the user want to properly use this application, and we leave the possibility to enrich informations about other important things of that mean of transport to the user it self.\\ 


\section{Stakeholders}
\begin{itemize}

\item (S01): Owing compaany - Obviously, the company itself is a stakeholder
\item (S02): External company - A general external company that provides means of transport or services (Google, ATM ...)

\end{itemize}

We can assume that another possible stakeholder could be the city (or region) itself because the local comunity can be interested in invest resources for reduce traffic and pollution.\\

At a higer level, for the same reasons , another possible stakeholder could be the governament (for the same reasons exposed before).

\section{Scope}

The applications requires a coverage of a city of a region but it doesn't specify which one so , we have to assume that this application should be able to manage his duty for any city or region of Italy (this is a limitation that we had to implement because it could be very hard to manage all public transport of the whole World).\\

The client's scope is tought as limited and personal , we haven't provided group solutions (like trips for schools or something like that).

\part{Overall Description}

\chapter{Domain Guide Lines}
In this chapter we will deline all those aspects that can be used to model our application for a future implementation

\section{Product Functions}
Here we include the most important requirements of our software

\begin{itemize}

\item (F01): The system must allow users to sign in
\item (F02): The system must allow the client to login in
\item (F03): The system must allow the client to log out
\item (F04): The system must allow the client to delete his account.
\item (F05): The system must provide an high level of security for client's data.
\item (F06): The software must be accessed on all majors mobile devices that runs Android Os, iOS and Windows Phone.
\item (F07): The software must interface with all major public transport companies that provides API.
\item (F08): The software must allow the user to define a journey.
\item (F09): The software must allow the user to define time constraint to a specific selected journey.
\item (F10): For each journey the software must provide all the available means of transport that can be used, according to the software interfaces.
\item (F11): The software must allow the user to select veichles that he/she wants to use for the journey.
\item (F12): The software must require the estimated time for each break of the journey.
\item (F13): If a journey is not possible because of possible overlaps with other journeys, the application must deny that option and notify the user.
\item (F14): If a journey is not possible because the required time to reach the destination is not enough, the application must deny that option and notify the user.
\item (F15): If a journey is not possible because a mean of transport selected by the user is not allowed to pass in a specific place, the application must deny that option and notify the user.
\item (F16): If a journey is not possible because breaks requires too much time, the application must deny that option and notify the user.
\item (F17): The software must provide all the available solutions, according to its setups, to carry the user from a place to another.
\item (F18): The software must aware the user about problems concerning the use of some means of transport (for strkes , road damages, rain ...) for which the usability is not guarantee.
\item (F19): The application should provide, as first option, the optimal solution according to the choices of the user (including preferences about pollution).
\item (F20): The application must be able to carry the user from the initial locality to the final one.
\item (F21): If a break runs out the time selected the application must notify the user.
\item (F22): The application should avoid to make the user pass trough danger zones of a city or region.
\item (F23): If the user has to pass trough a danger zone, the application must aware him.
\item (F24): In case of wearable devices setted , the application must notify the user also via that device.

\end{itemize}

\chapter{Specific Requirements}

\part{External Interface Requirements}

\section{User Interfaces}
The section above shows the main interfaces between the user and the app 
\begin{itemize}
\item (UI01): The user has to interface with the journey setter of the app
\item (UI02): The user has to interface with the navigator of the app.
\item (UI03): The user has to be allowed to interface with the technical support.
\item (UI04): The user has to be allowed to interface with the means of transport selection of the app.
\item (UI05): The user has to be allowed to interface with external companies book methods via the app.
\end{itemize}

And, more , the application must provide support for disable users like:
\begin{itemize}

\item (UI07):Support for users that have low level limitations and, because of this condition, cannot access the application in the standard way (like people with low vision disease). 
\item (UI08):Support for users that have high level limitation and , because of that, cannot use specific means of transport (like old people or people with movements limitation).

\end{itemize}

\section{Software Interface}
This section shows the main interfaces between the application and the software provided by the user/client's device.

\begin{itemize}
\item (SI01): The application must interface with the localization device disposed by the client/user's device.
\item (SI02): The application must interface with the system notification of the client/user's device.
\item (SI03): The application must interface with the standard I/O ways of interaction of the client/user's device.
\item (SI04): The application, in case of wearable device, must interface with the wearable device's notification system.
\item (SI05): The application should interface with the system watch of the client's device.
\item (SI06): The application should interface with the language selection of the system.
\item (SI07): The application must provide, in addition to the standard GUI, a low and high contrast mode. 
\item (SI08): The application must interface with external compalies API (those who are comtempled), in order to provide a way to access to transport's data.
\end{itemize}

\section{Hardware Interfaces}
This section shows the main interfaces between the software or client/user or application and the hardware of the client/user's device

\begin{itemize}
\item (HI01): The client's device must access to the Internet.
\item (HI02): The client's device must manage geolocalization. 
\end{itemize}

\section{Communication Interfaces}

\begin{itemize}
\item (MI01): The system must have a TCP/IP protocol.
\item (MI02): The system must have a way to manage the GSM protocol.
\item (MI03): In case of wearable devices , the system must have a way to manage the bluetooth protocol.
\item (MI04): The user device must have a way to manage 3G or wi-fi protocols.
\item (MI05): The user's device must have an I/O interface (like touch screens).
\item (MI06): The communication between application and the user must have a Restful way .
\end{itemize}

\chapter{Other Infos}

\part{Effort Spent}

\section{Bolshakova Liubov}
\begin{itemize}
\item (2017/10/08 - 7.00h) : Studyied the assignments, delined main parts , defined some Section 1,2 e 3 requirements.
\item (2017/10/12 - 1.00h) : Revision of the goals 
\item (2017/10/13 - 2.00h) : Improvement of the RASD, definition of group's standards and repository account redesign.
\end{itemize}

\section{Campagnoli Chiara}
\begin{itemize}
\item (2017/10/08 - 7.00h) : Studyied the assignments, delined main parts , defined some Section 1,2 e 3 requirements.
\item (2017/10/12 - 1.00h) : Revision of the goals 
\item (2017/10/13 - 2.00h) : Improvement of the RASD, definition of group's standards and repository account redesign.
\end{itemize}

\section{Lagni Luca}
\begin{itemize}
\item (2017/10/08 - 7.00h) : Studyied the assignments, delined main parts , defined some Section 1,2 e 3 requirements.
\item (2017/10/09 - 1.00h) : Created a tex RASD, writed down a first instance of section 1 and 2
\item (2017/10/11 - 1.30h) : writed down a fist instance of section 3 concerning instances
\item (2017/10/12 - 1.00h) : Revision of the goals 
\item (2017/10/13 - 2.00h) : Improvement of the RASD, definition of group's standards and repository account redesign.
\end{itemize}

\end{document}
